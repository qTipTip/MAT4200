\documentclass{article}

\usepackage[]{amsmath, amsthm, amssymb} 
\usepackage[]{microtype} 
\usepackage[]{enumerate}
\usepackage[shortlabels]{enumitem}
\usepackage[]{mathtools} 
\usepackage[usenames, dvipsnames]{color} 
\usepackage[]{tikz-cd} 
\setlist[description]{font=\normalfont\space}
\mathtoolsset{centercolon}
\newcommand{\Z}{\mathbb{Z}}
\newcommand{\m}{\mathfrak{m}}
\newcommand{\p}{\mathfrak{p}}
\newcommand{\q}{\mathfrak{q}}
\newcommand{\tensor}{\otimes}
\renewcommand{\a}{\mathfrak{a}}
\renewcommand{\b}{\mathfrak{b}}
\newcommand{\rad}[1]{\sqrt{#1}}
\DeclareMathOperator{\Ann}{Ann}
\DeclareMathOperator{\Ker}{Ker}

\title{\textsc{MAT4200 Fall 2016 \\
Mandatory Assignment}}
\author{Ivar Haugal{\o}kken Stangeby}
\date{\small\textbf{Due: Thursday, November 3, 2:00 PM}}

\begin{document}
    \maketitle     
    \section*{Problem 1}
    \label{sec:problem_1}
    Recall that an element $e$ in a ring $A$ is called \emph{idempotent} if it
    satisfies $e^2 = e$.
    \begin{enumerate}[(i)]
        \item  We wish to characterize the idempotents in the ring $\Z / (p^k)$
            given a prime $p$ and a positive integer $k$.  Note that any prime
            ideal in the ring $\Z / (p^k)$ is in one to one correspondence with
            prime ideals in $\Z$ containing $(p^k)$. There is only one maximal
            ideal in $\Z$ containing $(p^k)$, and that is $(p)$. Hence, $\Z /
            (p^k)$ is a local ring. Consequently, the only idempotent elements
            in $\Z / (p^k)$ is 0 and 1. This amounts to all elements $n \in \Z$
            such that $n \equiv 0 \pmod{p^k}$ or $n \equiv 1 \pmod{p^k}$.
        
        \item We can find the idempotents in $\Z / (12)$ by inspection. Since
            $\Z / (12) \simeq \Z_{12}$ the idempotent elements are $0, 1, 4$,
            and $9$. The recurring pattern seems to be, since $12 = 2^2\cdot
            3$, the idempotent elements are precisely those elements in $n \in
            \Z$ satisfying $n \equiv 0, 1 \pmod{2^2}$ \emph{and} $n \equiv 0, 1
            \pmod{3}$.

        \item We wish to find the number of idempotent elements in $\Z / (n)$
            where $n = \prod_{i=1}^{N} p_i^{n_i}$, $n_i \geq 1$ and the $p_i$
            are distinct prime numbers. Based on the observation from (ii) we
            conjecture that the idempotent elements in the ring $\Z / (n)$ are
            precisely those elements $m \in \Z$ satisfying $m \equiv 0, 1
            \pmod{p_i^{n_i}}$ for $i = 1, \ldots, N$. 

            The \emph{Chinese Remainder Theorem} tells us that if we have a set
            of pairwise coprime ideals, then
            \begin{equation}
                \notag
                \Z / ( \cap_{i = 1}^{N} (p_i^{n_i}) ) \simeq \prod_{i = 1}^{N} \Z / (p_i^{n_i}).
            \end{equation}
            Note however that the expression on the left involves an
            intersection of ideals, and not the product of ideals we have
            given. This can remedied by the fact that since the ideals are
            pairwise coprime, we have $\bigcap_{i=1}^N (p_i^{n_i}) =
            \prod_{i=1}^N (p_i^{n_i})$. For an element $(x_1, x_2, \ldots,
            x_N)$ to be idempotent in the expression on the right we need each
            $x_i$ idempotent in the corresponding ring. There are two choices
            of idempotent element for each $x_i$, and there are $N$ choices to
            be made, so we have $2^N$ idempotent elements in total. Hence,
            following the isomorphisms, we can conclude that $\Z / (n)$
            contains $2^N$ idempotent elements.
\end{enumerate} 

    \section*{Problem 2}
    \label{sec:problem_2}
        
    We let $A$ be a ring, $\mathfrak{a}$, $\mathfrak{b}$ ideals in $A$, and
    $\mathfrak{p}$ a prime ideal. We wish to show that the following statements
    are equivalent:
    \begin{enumerate}[(1)]
        \item $\a \subseteq \p$ or $\b \subseteq \p$;
        \item $\a \cap \b \subseteq \b$;
        \item $\a\b \subseteq \p$.
    \end{enumerate}
    \begin{proof} 
        \hfill
        \begin{description}
            \item[(1) $\implies$ (2):]\hfill  \\
                Let $x \in \a\cap\b$.
                Then $x$ is an element of both $\mathfrak{a}$ and
                $\b$. If $\a \subseteq \p$, then
                $x \in \p$. Similarly, if $\b \subseteq
                \p$, then $x \in \p$. In an case,
                $\a\cap\b \subseteq \p$.
            \item[(2) $\implies$ (3):]\hfill \\
                Assume that $\a \cap \b \subseteq \p$. Let $x \in \a\b$. We
                wish to show that $x \in \p$. By definition of $\a\b$ we know
                that $x = \sum^{}_{} a_ib_i$ where each $a_i \in \a$ and $b_i
                \in \b$. Note that each term $a_ib_i$ is in both $\a$ and $\b$
                as these are ideals. So by assumption, $a_ib_i \in \p$. Since
                we have a sum of elements in $\p$, it must also lie in $\p$.
                Hence $x \in \p$.
            \item[(3) $\implies$ (1):]\hfill \\
                We prove this contrapositively. Assume that $\a \not\subseteq
                \p$, $\b \not\subseteq \p$ and choose $x \in \a \smallsetminus
                \p$, $y \in \b \smallsetminus \p$. Since $\p$ is prime we can
                use the fact that $x, y \notin \p \implies xy \notin \p$. Now,
                $xy$ is an element in $\a\b$, but $xy \notin \p$, hence $\a\b
                \not\subseteq \p$. We have now shown contrapositively that (3)
                $\implies$ (1).
        \end{description}
    \end{proof}
    
    \section*{Problem 3}
    \label{sec:problem_3}
     
    For this problem we assume that $A$ is a ring, $B$ an algebra with the
    structure map $f \colon A \to B$\footnote{I made a slight change in
    names, calling $A'$ for $B$ instead. Due to simplicity.}. Furthermore,
    we assume $B$ to be flat as an $A$-module. We recall that $B$ is
    \emph{faithfully flat} if for every $A$-module $M$, the map $M \to M
    \tensor_A B$ given by $x \mapsto x \tensor 1$ is injective. We wish to
    show that the following statements are equivalent:

    \begin{enumerate}[(1)]
        \item $B$ is faithfully flat;
        \item every ideal of $A$ is the contraction of its extension, i.e.,
            $f^{-1}(\a B) = \a$ for every ideal $\a \subseteq A$;
        \item every prime ideal of $A$ is the contraction of a prime ideal of
            $B$;
        \item for every maximal ideal $\m \subset A$, the ideal $\m B$ is
            different from $B$;
        \item for any nonzero $A$-module $M$, the module $M \tensor_A B$ is
            nonzero.
    \end{enumerate}

    \begin{proof} The plan for the proof is as follows:
        \begin{equation}
            \notag
            \begin{tikzcd}
                (1) \ar[r, Rightarrow] & (2) \ar[r, Rightarrow] & (3) \ar[r, Rightarrow] & (4) \ar[r, Rightarrow] & (5) \ar[llll, bend left, Rightarrow]\\
            \end{tikzcd}
        \end{equation}

        \begin{description}
            \item[(1) $\implies$ (2):]\hfill \\ We assume $B$ to be
                faithfully flat. Let $g\colon \a \to \a \tensor_A B$ be
                the map defined by $x \mapsto x \tensor 1$.  By
                assumption, since $B$ is faithfully flat, $g$ is
                injective. We also have an isomorphism $\varphi \colon \a
                \tensor B \to \a B$ given by $a\tensor b \mapsto ab$. Note
                that the multiplication takes place in $B$, so by
                definition, we have $ab \coloneqq f(a)b$ as the only
                sensible thing to do. We are interested in the function $h
                \coloneqq \varphi \circ g$ taking elements in $\a$ to
                elements in $\a B$ as the following diagram illustrates:

            \begin{equation}
                \notag
                \begin{tikzcd}
                    \a \ar[r, "g"] \ar[dr, "h"] & \a \tensor_A B \ar[d, "\varphi"]\\
                    & \a B
                \end{tikzcd}
            \end{equation}

            Taking $a \in \a$ we see that $h(a) = \varphi(g(a)) = \varphi(a
            \tensor 1) = f(a)$. So $f$ factors through $\varphi$ and $g$. We
            can therefore consider the contraction of $\a B$ as passing through
            $g$ and $\varphi$. Consider now $f^{-1}(\a B) = g^{-1}\circ
            \varphi^{-1} (\a B)$. Now, $\varphi$ is an isomorphism so this
            equals $g^{-1}(\a \tensor_A B)$ and $g$ is injective, so we get
            $f^{-1}(\a B) = \a$.
    
            \item[(2) $\implies$ (3):]\hfill \\
                Assume that $f^{-1}(\a B) = \a$ for every ideal $\a$ in $A$.
                Let $\p$ be a prime ideal in $A$. By assumption, we know that
                $\p = f^{-1}(\p B)$. By Proposition 3.16, $\p$ is the
                contraction of a prime ideal of $B$ if and only if $\p^{ec} =
                \p$.

            \item[(3) $\implies$ (4):]\hfill \\
                Assume that every prime ideal of $A$ is the contraction of a
                prime ideal of $B$. Let $\m$ be a maximal ideal of $A$. So, we
                have $\m = \q^c$ for some prime ideal $\q$ in $B$. We need to
                show that $\m^e \neq B$. Extending, we see that $\m^e =
                \q^{ce}$ but $\q^{ce} \subseteq \q$. Since $\q$ is prime, it is
                properly contained in $B$, hence $\m^e$ is properly contained
                in $B$. This proves the claim.

            \item [(4) $\implies$ (5):]\hfill \\
                Assume that for every maximal ideal $\m$ we have $\m B \neq B$.
                Assume that $M \neq 0$. We wish to show that $M \tensor_A B
                \neq 0$. We do this by constructing a non-zero submodule of $M
                \tensor_A B$. Since $M \neq 0$ we chose $x \neq 0$ as an
                element in $M$. We know that the annihilator $\a \coloneqq
                \Ann(x)$ has to be a proper ideal of $A$, as $1 \notin \a$.
                The map $\varphi\colon A \to M$ given by $a \mapsto ax$ have
                precisely $\a$ as its kernel. This induces an injection from
                $(A / \a)$ to $M$. By flatness of $B$, we can tensor and
                preserve injectiveness as the following diagram illustrates:
                \begin{equation}
                    \notag
                    \begin{tikzcd}
                        (A / \a) \tensor_A B \ar[r, hookrightarrow] \ar[d, "\simeq"] & M \tensor_A B \\
                        B / \a B \ar[ur, hookrightarrow]
                    \end{tikzcd}
                \end{equation}
                Since $\a$ is an ideal of $A$ and contained in a maximal ideal
                $\m$, we have that $\a B \subseteq \m B \subset B$, hence $B /
                \a B \neq 0$, which tells us that $M \tensor_A B \neq 0$.

            \item [(5) $\implies$ (1):]\hfill\\
                Assume the contrapositive of the hypothesis, namely that
                $M\tensor_A B = 0 \implies M = 0$. Fix the $A$-module $M$ and
                let $g: M \to M \tensor B$ given by $x \mapsto x \tensor 1$ be
                the canonical map. We need to show that $\Ker(g) = 0$.  Since
                $g$ is an $A$-module homomorphism, $\Ker(g)$ is an $A$ module.
                We tensor it with $B$ to to obtain $\Ker(g) \tensor_A B$. Let
                $x \tensor b$ be an element. Since $M \tensor_A B$ is naturally
                a $B$-module we can write this as $b(x \tensor 1)$, but our $x$
                lies in $\Ker(g)$ so, $x \tensor 1 = 0$.  Consequently,
                $\Ker(g) \tensor_A B = 0$. By our assumption, we must have
                $\Ker(g) = 0$, so $g$ is injective. Since $M$ was arbitrary,
                $B$ must be faithfully flat.
        \end{description}
    \end{proof}

\section*{Problem 4}
\label{sec:problem_4}

Let $k$ be a field, and let $A \coloneqq k[x, y, z]$ be the polynomial
ring in three variables. Set $\a \coloneqq (xy, x-yz)$, $\q_1 \coloneqq
(x, z)$, and $\q_2 \coloneqq (y^2, x-yz)$. We wish to show that $\a = \q_1
\cap \q_2$ and that this is a minimal primary decomposition of $\a$.
\begin{proof}
    We start by showing the inclusion $\q_1 \cap \q_2 \subseteq \a$. Let $f \in
    \q_1 \cap \q_2$. We can then write $f = ax + az = cy^2 + d(x - yz)$ for $a,
    b, c, d \in A$.  Rearranging we see that $x(a - d) = cy^2 - bz - dyz$. From
    this equality we can conclude that $x | cy^2 - bz - dyz$. In particular,
    $x$ divides $c$, so $c = c'x$ for some $c' \in A$. Substituting, we can
    write $f = c'xy^2 + d(x - yz)$ which is a linear combination of the
    generators of $\a$. We can therefore conclude that $f \in \a$.

    Conversely, to show $\a \subseteq \q_1 \cap \q_2$ we can consider the
    generators of $\a$. If we can show that they lie in the intersection, then
    so must any linear combination of the two. Consider first the generator
    $xy$ of $\a$. We see immediately that $xy \in \q_1$. Similarly, the
    generator $x - yz$ is a linear combination of $x$ and $z$, hence lies in
    $\q_1$. Since it also lies in $\q_2$ as a generator, it remains to show
    that $xy$ lies in $\q_2$. In other words, we need to write $xy = ay^2 + b(x
    - yz)$ for some $a, b \in A$. Taking $a = z$ and $b = y$ we get $zy^2 + yx
    - y^2z$ which equals $xy$. Hence $xy \in \q_2$. Since both generators of
    $\a$ lie in $\q_1 \cap \q_2$ we can conclude that $\a \subseteq \q_1 \cap
    \q_2$.

    We now show that both $\q_1$ and $\q_2$ are primary ideals. Consider the
    quotient $A / \q_1$. This is isomorphic to $k[y]$, and since $k$ is a
    field, this is an integral domain. Hence $k[y]$ contains no zero divisors,
    so vacuously, all zero divisors are nilpotent. Hence $\q_1$ is primary.
    Equivalently, note that modding out by $\q_1$ yields an integral domain,
    hence $\q_1$ is prime and any prime ideal is primary.
    
    Consider now the quotient $A / \q_2$. Let $\bar{x}, \bar{y}$ and $\bar{z}$
    denote the equivalence classes. Modding $A$ out by $(x-yz)$ yields
    $k[\bar{y}\bar{z}, \bar{y}, \bar{z}]$. This ring is equal to $k[\bar{y},
    \bar{z}]$. Modding out by $(y^2)$ yields $A / \q_2 = k[\bar{y}, \bar{z}] /
    (y^2)$. It remains to show that all zero-divisors here are nilpotent
    elements. The only way for two non-zero elements $f, g \in A/\q_2$ to
    multiply to $0$ is if $y$ divides both $f$ and $g$. In other words, both
    $f$ and $g$ are nilpotent, hence $\q_2$ is a primary ideal.
    
    Furthermore, we have that $y^2 \in \q_1$ but $y^2 \notin \q_1$ and $z \in
    \q_1$ but $z \notin \q_2$. The radicals are distinct, as
    \begin{align*}
        \notag
        \rad{(x, z)} = (x,z) \text{ and } \rad{(y^2, x - yz)} = (y, x - yz),
    \end{align*}
    hence $\q_1\cap\q_2$ is a minimal primary decomposition of $\a$.
\end{proof}

\section*{Problem 5}
We let $k$ be a field, and $X, Y, Z$ variables. We set
\begin{equation}
    \notag
    A \coloneqq k[X, Y, Z] / (X^2- Y^3 - 1, XZ - 1), 
\end{equation}
and let $x, y, z \in A$ be the classes of $X, Y, Z$ in $A$. Fixing $a, b \in k$
we set $t \coloneqq x + ay + bz$ and $B \coloneqq k[t] \subseteq A$. We wish to
show that $x$ and $y$ are integral over $B$ for any $a, b$, and that $z$ is
integral over $B$ if and only if $b \neq 0$.

Note first, that in the quotient $A$, we have that $xz = 1$ and that $y^3 = x^2
- 1$. Our goal is to manipulate the expression for $t$ in such a way that we
get rid of $y$ and $z$ and end up with an integral dependence relation for $x$
over $B$. Multiplying the equation for $t$ by $x$ yields $tx = x^2 + ayx + b$
where the $z$ is gone. Rearrange to get $x^2 + tx + b = ayx$. We now need to
get rid of $y$, and we do this by cubing both sides. This gives us
\begin{equation}
    \notag
    (x^2 - tx + b)^3 + a^3x^3(x^2 - 1) = 0
\end{equation}
which we recognize as an integral dependence relation for $x$ over $B$. Note
that all coefficients are in $B$, namely $a, b$ and $t$. If we now can show
that $y$ is integral over $B[x]$, then it follows by transistivity that $y$ is
integral over $B$. Since $y^3 = x^2 - 1$, we can rearrange to get an integral
dependence relation for $y$ over $B[x]$, namely $y^3 - x^2 + 1 = 0$. So, $y$ is
integral over $B[x]$, and therefore also integral over $B$.

To show that $z$ is integral over $B$ if and only if $b \neq 0$, we first
assume that $b\neq 0$. We then get the integral dependence relation for $z$
over $B[x, y]$, namely $z + (a y + x - t) / b = 0$. By transitivity, $z$ is
integral over $B$. Conversely, to show that $z$ integral over $B$ implies $b
\neq 0$, we assume for contradiction that $z$ is integral over $B$ and that $b
= 0$. The contradiction we want relies on the fact that $k[x]$ is integrally
closed in its field of fractions $k(x)$. We wish to show that $z$ is an element
of $k[x]$ by showing that $z$ is integral over $k[x]$. Note first that $z \in
k(x)$ as $z = 1 / x$. 

With $a$ arbitrary and $b = 0$, $t$ reduces to $x + ay$, hence our ring $B$ is
contained in the ring $k[x, y]$. Under the assumption that $z$ is integral over
$B$, it must also be integral over $k[x, y]$ using the same dependence
relation. Futhermore, $y$ is integral over $k[x]$ as $y^3 - x^2 + 1 = 0$.
Consequently, $z$ is integral over $k[x]$. But, now, since $z \in k(x)$ and $z$
integral over $k[x]$, we must have $z \in k[x]$. This means that $1 / x \in
k[x]$, which is the contradiction we wanted. Hence $z$ is integral over $B$ if
and only if $b \neq 0$.
 
\end{document}
