\documentclass{article}

\usepackage[]{amsmath, amsthm, amssymb} 
\usepackage[]{faktor} 
\usepackage[]{cleveref} 
\usepackage[]{color} 
\usepackage[]{enumerate}
\usepackage[]{tikz-cd} 

\usepackage[margin=1.5in]{geometry} 
\usepackage[]{microtype} 

\theoremstyle{definition}
\newtheorem{prop}{Proposition}
\newtheorem{claim}{Claim}
\newtheorem*{exercise}{Exercise}
\newtheorem*{problem}{Problem}

\DeclareMathOperator{\im}{im}
\DeclareMathOperator{\id}{id}
\DeclareMathOperator{\coker}{coker}

\newcommand{\loc}[2]{#1^{-1}#2}
\renewcommand{\a}{\mathfrak{a}}
\renewcommand{\p}{\mathfrak{p}}
\renewcommand{\m}{\mathfrak{m}}
\renewcommand{\q}{\mathfrak{q}}
\renewcommand{\b}{\mathfrak{b}}
\renewcommand{\rad}[1]{\sqrt{#1}}
\renewcommand{\Z}{\mathbb{Z}}
% I want unnumbered sections, while still showing up in table of contents.
\setcounter{secnumdepth}{0}

\title{\textsc{Commutative Algebra} \\
Notes in MAT4200}
\author{Ivar Stangeby}
\date{\today}

\begin{document}

\maketitle

\begin{abstract}
This document contains attempted solutions to the exercises given in the book
\emph{Introduction to Commutative Algebra} by M. F. Atiyah and I. G. McDonald.
\textcolor{red}{Text highlighted in red} is meant to signify logical passages
where I feel I have no idea what I am doing, and the reason for leaving this in
is to later be able to reflect on the thought process.
\end{abstract}
\tableofcontents

\section{Exercises}
\label{sec:exercises_chapter_1}

\subsection{Chapter 1: Rings and ideals}
\label{sub:chapter_1}

\begin{exercise}[1]
    Assume that $x$ is nilpotent, and that $1 + x$ is \emph{not} a
    unit in $A$. Hence, $1 + x$ is contained in a maximal ideal
    $\mathfrak{m}$. Since any maximal ideal is prime, and $x$ is
    nilpotent, we have $x^n = 0 \in \mathfrak{m} \implies x \in
    \mathfrak{m}$. Any ideal is an additive subgroup, so $1 \in
    \mathfrak{m}$ which contradicts the fact that $\mathfrak{m}$ is
    maximal.

    Now assume $u$ a unit and $x$ nilpotent. Assume for the sake of
    contradiction that $u + x$ is \emph{not} a unit in $A$. Then $u +
    x$ is contained in a maximal ideal $\mathfrak{m}$. Since $x$ is
    nilpotent we have $x \in \mathfrak{m}$, hence $u \in \mathfrak{m}$
    so $\mathfrak{m} = (1)$, again contradicting the fact that
    $\mathfrak{m}$ is maximal.
\end{exercise}

\begin{exercise}[4]
    We want to show that in $A[x]$ we have $\mathfrak{N} =
    \mathfrak{R}$. We have trivially that $\mathfrak{N} \subseteq
    \mathfrak{R}$, so we only need to show the opposite inclusion.

    Let $f \in \mathfrak{R}$ with $f = \sum^{n}_{i=0} a_ix^i$, so by
    proposition 1.9 we have $1 - fg$ a unit for all $g \in A[x]$. Let
    $g = x$ be an element in $A[x]$. Then the function
    \begin{equation}
        \notag
        1 - a_0x - a_1x^2 - \ldots - a_nx^{n+1}
    \end{equation}
    is a unit in $A[x]$. By exercise 1.2.(i) we have that $a_0, \ldots
    ,a_n$ are nilpotent in $A$. By exercise 1.2.(ii) we have that $f$
    is nilpotent, so $f \in \mathfrak{N}$. Hence $\mathfrak{N} =
    \mathfrak{R}$.
\end{exercise}

\begin{exercise}[6]
    Let $A$ be a ring such that any ideal not contained in
    $\mathfrak{N}$ contains a non-zero idempotent element. We want to
    show that the nilradical and the Jacobson radical coincide in this
    case. We have the inclusion $\mathfrak{N} \subseteq \mathfrak{R}$
    trivially. For the opposite inclusion we argue contrapositively.
    Let $c \notin \mathfrak{N}$. Then $(c) \not \subseteq
    \mathfrak{N}$. By assumption, $(c)$ contains an idempotent element
    $a = cx$ for some $x \in A$. We wish to use proposition 1.9 again.
    Consider the element $1 - a$, and note that $a(1 - a) = a - a =
    0$, so $1 - a$ is \emph{not} a unit in $A$ since it is a zero
    divisor. By proposition 1.9 we have $a \notin \mathfrak{N}$, so
    $(c) \not \subseteq \mathfrak{N}$. Consequently, $\mathfrak{R}
    \subseteq \mathfrak{N}$.
\end{exercise}

\begin{exercise}[7] Let $A$ be a ring in which every element satisfies
    $x^n = x$ for some $n \geq 2$ dependent on $x$. We want to show
    that the nilradical $\mathfrak{N}$ and the Jacobson radical
    $\mathfrak{R}$ coincide.  The inclusion $\mathfrak{N} \subseteq
    \mathfrak{R}$ is trivial as any maximal ideal is prime. We show
    the opposite inclusion by a contrapositive argument.

    Assume that $x \notin \mathfrak{N}$. Our plan is to show that $1 -
    xg$ is \emph{not} a unit for any $g \in A$. Consider the element
    $1 - x \cdot x^{n-2}$. This is a zero divisor as shown by
    multiplying by $x$ from the left. Hence $1 - xg$ is \emph{not} a
    unit with $g = x^{n-2}$. By proposition 1.9 we then have $x \notin
    \mathfrak{R}$. This shows contrapositively that $\mathfrak{R} =
    \mathfrak{N}$.

    \textcolor{red}{Note that I did not prove that every prime ideal
        $\mathfrak{p}$ is also a maximal ideal as the exercise requested.
    Trying again below.}

    We seek to show that any prime ideal $\mathfrak{p}$ of $A$ is
    maximal. Let $\varphi : A \to A / \mathfrak{p}$ be the canonical
    homomorphism. Let $x \in A$ be any element. We know that $x^n = x$
    for some $n$. Let $\bar{x} = \varphi(x) \in A / \mathfrak{p}$.
    Note that $\bar{x} = \bar{x}^n$. Since $A / \mathfrak{p}$ is an
    integral domain due to $\mathfrak{p}$ being prime, we know that
    the cancellation law for multiplication holds. So $\bar{x}^n = 1
    x$ implies that $\bar{x}^{n-1} = 1$. So, $\bar{x}$ has
    inverse $\bar{x}^{n-2}$. Hence, $A / \mathfrak{p}$ is a field,
    which implies that $\mathfrak{p}$ is a maximal ideal.
\end{exercise}

\begin{exercise}[8] Let $A$ be a non-zero ring. We wish to show that the
    set of prime ideals of $A$ has a minimal element with respect to
    set inclusion.  This can be solved by an application of Zorn's
    Lemma.  Let $\Sigma$ denote the set of all prime ideals of $A$.
    Let $\mathfrak{q} \leq \mathfrak{p}$ if $\mathfrak{p} \subseteq
    \mathfrak{q}$ (note the reverse inclusion).  This set is partially
    ordered with respect to this relation. We need to show that any
    chain $\Gamma$ in $\Sigma$ has an upper bound in $\Sigma$.  Let
    $\mathfrak{P} = \bigcap_{\alpha} \mathfrak{p}_\alpha$. We claim
    that the intersection of all prime ideals $\mathfrak{P}$ is an
    element of $\Sigma$ and an upper bound for $\Gamma$.

    Let $xy \in \mathfrak{P}$. Then $xy$ is an element of every prime
    ideal $\mathfrak{p}_\alpha$ in the chain $\Gamma$.  Assume now
    that $x \notin \mathfrak{P}$. In this case we need to show that $y
    \in \mathfrak{P}$. Let $\mathfrak{p}_i$ be a prime ideal not
    containing $x$. It therefore contains $y$ instead. Since $\Gamma$
    is totally ordered, we can consider all the elements
    $\mathfrak{p}_i \leq \mathfrak{p}'$, i.e., $\mathfrak{p}'
    \subseteq \mathfrak{p}_i$. Since $x \notin \mathfrak{p}_i$ we have
    $x \notin \mathfrak{p}'$. It then follows that all such
    $\mathfrak{p}'$ must contain $y$. 

    Consider now the elements $\mathfrak{p}' \geq \mathfrak{p}_i$,
    that is $\mathfrak{p}_i \subseteq \mathfrak{p}'$. Since $y \in
    \mathfrak{p}_i$ we have $y \in \mathfrak{p}'$. Consequently, we
    have $y$ in all prime ideals, hence also in $\mathfrak{P}$. This
    shows that $\mathfrak{P}$ is an element of $\Sigma$.

    To show that $\mathfrak{P}$ is an upper bound for $\Gamma$, let $I \in
    \Gamma$ be a prime ideal. Then by definition of $\mathfrak{P}$ we have
    $\mathfrak{P} \subseteq I$, hence $I \leq \mathfrak{P}$.  Zorn's Lemma
    then guarantees the existence of a maximal element with respect the
    order $\leq$, and consequently we have shown the existence of a minimal
    element with respect to set inclusion.
\end{exercise}  

\begin{exercise}[9]
    Assume that $\mathfrak{a} = r(\mathfrak{a})$. By proposition 1.14 the
    radical of any ideal $\mathfrak{a}$ is the intersection of the prime
    ideals containing $\mathfrak{a}$. It therefore directly follows from
    our assumption that $\mathfrak{a}$ is an intersection of prime ideals.
    Assume now that $\mathfrak{a}$ is \emph{not} an intersection of prime
    ideals. Then it cannot be equal to the radical, as the radical
    \emph{is} an intersection of prime ideals.
\end{exercise}

\begin{exercise}[10]
    Let $A$ be a ring, and $\mathfrak{N}$ its nilradical. We wish to show that the following are equivalent:
    \begin{enumerate}[(i)]
        \item $A$ has exactly one prime ideal;
        \item every element of $A$ is either a unit or nilpotent;
        \item $A / \mathfrak{N}$ is a field.
    \end{enumerate}
    We first show (i) $\implies$ (ii). Assume that $A$ has exactly one
    prime ideal. Let $x$ be an element in $\mathfrak{N}$. In that case it
    is nilpotent. Assume therefore that $x$ is not an element in
    $\mathfrak{N}$. If we further assume that $x$ is \emph{not} a unit,
    then it is contained in a maximal ideal $\mathfrak{m}$. Since $A$ has
    exactly one prime ideal, and any maximal ideal is prime, we must have
    $\mathfrak{m} = \mathfrak{N}$. This contradicts the fact that $x \notin
    \mathfrak{N}$. Consequently, $x$ must be a unit. So any element $x$ in
    $A$ is either nilpotent, or a unit.

    We now consider the implication (ii) $\implies$ (iii). Assume that
    every element of $A$ is either a unit or nilpotent. We seek to show
    that $A / \mathfrak{N}$ is a field. In principle, we only need to show
    that $\mathfrak{N}$ is in fact a maximal ideal. Let $\mathfrak{a}$ be
    an ideal containing $\mathfrak{N}$. We need to show that $\mathfrak{a}
    = \mathfrak{N}$, or $\mathfrak{a} = A$.

    If $\mathfrak{a} = \mathfrak{N}$, then we are done. Assume therefore
    that $\mathfrak{a} \neq \mathfrak{N}$. Then there is an element $x \in
    \mathfrak{N}$ that is not in $\mathfrak{a}$. Then $x$ is \emph{not} a
    nilpotent element, so by assumption it must be a unit. Since
    $\mathfrak{a}$ is an ideal containing a unit, we must have
    $\mathfrak{a} = A$. Hence $\mathfrak{N}$ is maximal and $A /
    \mathfrak{N}$ is a field.

    We now show the final implication (iii) $\implies$ (i). Assume that $A
    / \mathfrak{N}$ is a field. Then $\mathfrak{N}$ is a maximal ideal in
    $A$. Let $\mathfrak{p}$ be a prime ideal of $A$. Then $\mathfrak{N}
    \subseteq \mathfrak{p}$. If $\mathfrak{N} = \mathfrak{p}$ we are done.
    If not, since $\mathfrak{N}$ is maximal, we must have $\mathfrak{p} =
    A$, hence $A$ contains \emph{exactly} one prime ideal.
\end{exercise}

\begin{exercise}[11]
    Let $A$ be a boolean ring (i.e., $x^2 = x$ for all $x \in A$). We want
    to show that the following properties hold:
    \begin{enumerate}[(i)]
        \item $2x = 0$ for all $x \in A$;
        \item every prime ideal $\mathfrak{p}$ in $A$ is maximal, and
            $A / \mathfrak{p}$ is a field with two elements; and
        \item every finitely generated ideal in $A$ is principal.
    \end{enumerate}
    For (i), let $x$ be an element in $A$ and let $a$ be the additive
    inverse of $x$. That is $a + x = x + a = 0$. Multiplying both sides by
    $x - a$ yield
    \begin{equation}
        \notag
        x^2 - a^2 = 0 \iff x - a = 0 \iff x = a.
    \end{equation}
    It then follows that $2x = x + x = x + a = 0$.

    For (ii), note that this is just a special case of exercise 7 with $n =
    2$ for every $x \in A$, hence any prime ideal $\mathfrak{p}$ is also
    maximal. It remains to show that $A / \mathfrak{p}$ has two elements.
    Let $x$ be an element of $A$ and assume that $x \in \mathfrak{p}$. Then
    $\varphi(x) = 0$ in $A / \mathfrak{p}$. If $x \notin \mathfrak{p}$ then
    $\varphi{x}$ has an inverse, so it makes sense to look at
    $\varphi(x)\varphi(x^{-1}) = 1$. Multiplying by $\varphi(x)$ on both
    sides yields
    \begin{equation}
        \notag
        \varphi(x^2)\varphi(x^{-1}) = \varphi(x)\varphi(x^{-1}) = 1 = \varphi(x).
    \end{equation}
    So $\varphi(x) = 1$ in $A / \mathfrak{p}$. Hence, $A / \mathfrak{p} =
    \left\{0, 1\right\}$, namely the additive and the multiplicative
    identities.
\end{exercise}

\begin{exercise}[12]
    We wish to show that a local ring $A$ has no idempotent element not
    equal to $0$ or $1$. So, let $x \in A$ be idempotent with $x \neq 0,
    1$. We consider two cases --- assume first that $x$ is a unit. But
    then, we have $x = x^{-1}x^2 = x^{-1}x = 1$ which contradicts our
    initial assumption.

    Assume therefore that $x$ is \emph{not} a unit. By proposition
    1.5, we must have $x$ contained in some maximal ideal
    $\mathfrak{m}$. Since $A$ is local, there is only one maximal
    ideal, hence $x \in \mathfrak{m} \implies x \in \mathfrak{R}$.
    Since $x$ is in the Jacobson radical, we know that $1 - xy$ is a
    unit in $A$ for all $y$ in $A$. Consider the fact that $x$ is
    idempotent, so
    \begin{equation}
        \notag
        x^2 = x \implies x(1 - x) = 0.
    \end{equation}
    Since $x \neq 0, 1$ we have that $(1 - x)$ is a zero divisor in $A$,
    contradicting the fact that it is also a unit.\footnote{Equivalently, since
        both $x$ and $(1 - x)$ are non-units, they must both lie in a maximal
        ideal. Since $A$ is local they lie in the same maximal ideal. Since any
    ideal is an additive subgroup we have that $x + (1 - x) = 1$ lie in the
maximal ideal, hence the maximal ideal is the whole thing, $(1)$, which is a
contradiction.} Consequently, if $x$ is idempotent, it must be either $0$ or
$1$.
\end{exercise}
\begin{exercise}[15. The prime spectrum of a ring]
    Let $A$ be a ring, and $X$ the set of all prime ideals of $A$. For
    each subset $E$ of $A$, define $V(E)$ to be the set of prime
    ideals of $A$ containing $E$. We want to prove that the following holds:
    \begin{enumerate}[(i)]
        \item if $\mathfrak{a}$ is the ideal generated by $E$,
            then $V(E) = V(\mathfrak{a}) =
            V(\sqrt{\mathfrak{a}})$;
        \item $V(0) = X$, $V(1) = \emptyset$;
        \item if $\left\{ E_i \right\}_{i \in I}$ is any family of
            subsets of $A$, then
            \begin{equation}
                \notag
                V\left(\bigcup_{i \in I} E_i\right) =
                \bigcap_{i \in I} V(E_i);
            \end{equation}
        \item $V(\mathfrak{a} \cap \mathfrak{b}) =
            V(\mathfrak{a}\mathfrak{b}) = V(\mathfrak{a})\cup
            V(\mathfrak{b})$.
    \end{enumerate}
    For (i), assume that $\mathfrak{a}$ is generated by $E$. That is
    \begin{equation}
        \label{eq:gen_ideal}
        \tag{*}
        \mathfrak{a} = \left\{ \sum_{i\in I} a_ie_i \mid a_i \in A, e_i \in E \right\}.
    \end{equation}
    Let $\mathfrak{p} \in V(E)$ be a prime ideal containing $E$. Since
    any ideal is an abelian additive group closed under multiplication
    from $A$ we see that any product $a_ie_i$ is in $\mathfrak{p}$ for
    any $a_i \in A, e_i \in E \subseteq \mathfrak{p}$. Since
    $\mathfrak{p}$ is an additive group, it follows that any sum $
    \sum^{}_{i \in I} a_i e_i$ is also in $\mathfrak{p}$. So, all the
    elements in $\mathfrak{a}$ are in $\mathfrak{p}$ hence
    $\mathfrak{p} \supseteq \mathfrak{a}$ and consequently
    $\mathfrak{p} \in V(\mathfrak{a})$. Conversely, assume that
    $\mathfrak{p} \in V(\mathfrak{a})$. Since $\mathfrak{p}$ contains
    $\mathfrak{a}$ it must also contain any elements on the form
    $\sum^{}_{i \in I} e_i$, so it contains $E$; set $a_i = 1$ for all
    $i$ in \cref{eq:gen_ideal}. We have the first equality, $V(E) =
    V(\mathfrak{a})$.

    Now, let $\mathfrak{p} \in V(\mathfrak{a})$ so $\mathfrak{p}$
    contains $\mathfrak{a}$. Consequently $\mathfrak{p} \supseteq
    \bigcup_{i \in I}\mathfrak{p}_i$ where each $\mathfrak{p}_i$
    contain $\mathfrak{a}$. But by proposition 1.14, it follows that
    $\mathfrak{p}_i$ contain $\sqrt{\mathfrak{a}}$, so $\mathfrak{p}
    \in V(\sqrt{\mathfrak{a}})$. Conversely, let $\mathfrak{p} \in
    V(\sqrt{\mathfrak{a}})$. Then $\mathfrak{p}$ contains
    $\sqrt{\mathfrak{a}}$. Since the radical of an ideal contains the
    ideal, it follows that $\mathfrak{p}$ contains $\mathfrak{a}$.
    Hence $\mathfrak{p} \in V(\mathfrak{a})$. This yields the final
    equality: $V(\mathfrak{a}) = V(\sqrt{\mathfrak{a}})$.

    For (ii), note that any prime ideal $\mathfrak{p}$ contains the
    zero ideal, so $V(0) = X$. Similarly, by definition, a prime ideal
    is not equal to $(1)$, so $V(1) = \emptyset$.

    For (iii), let $\mathfrak{p} \in V\left(\bigcup_{i \in I}
    E_i\right)$. This means that $\mathfrak{p}$ contains $E_i$ for
    each $i \in I$, so $\mathfrak{p} \in V(E_i)$ for each $i \in I$.
    It then follows that $\mathfrak{p} \in \bigcup_{i \in I} V(E_i)$.
    Conversely, let $\mathfrak{p} \in \bigcup_{i \in I} V(E_i)$. This
    means that $\mathfrak{p}$ contains $E_i$ for each $i \in I$.
    Consequently, $\mathfrak{p}$ contains the union of all $E_i$,
    hence $\mathfrak{p} \in V\left(\bigcup_{i \in I} E_i \right)$.
\end{exercise}

\subsection{Chapter 2: Modules}
\label{sec:chapter_2}

\begin{exercise}[1]
    We want to show that $\left( \mathbb{Z}/m\mathbb{Z}
    \right)\otimes_{\mathbb{Z}}  \left( \mathbb{Z} / n\mathbb{Z} \right) =
    0$ if $m, n$ are coprime. By definition of coprime, we have $\gcd(m, n)
    = 1$. We can find numbers $a, b \in \mathbb{Z}$ such that $am + bn =
    1$. Let $x \otimes y$ be an element of the tensor product. We therefore
    have
    \begin{align*}
        x \otimes y = 1 \left( x \otimes y \right) = (am + bn)(x
        \otimes y) = a(mx \otimes y) + b(x \otimes ny) = 0.
    \end{align*}
\end{exercise}

\begin{exercise}[2]
    We let $A$ be a ring, $\mathfrak{a}$ an ideal, and $M$ an $A$-module. We
    wish to show that $(A / \mathfrak{a}) \otimes_A M \simeq M /
    \mathfrak{a}M$. We right tensor the exact sequence
    \begin{equation}
        \notag
        0 \to \mathfrak{a} \to A \to A / \mathfrak{a} \to 0
    \end{equation}
    with $M$. This gives another exact sequence (proposition 2.18)
    \[
    \begin{tikzcd}
        \mathfrak{a} \otimes M \arrow{r}{f} \arrow{dr} & A \otimes M \arrow{r}{g} \arrow{d}{\psi} &(A / \mathfrak{a}) \otimes M \arrow{r} & 0 \\
        & M
    \end{tikzcd}
    \]
    where $\psi$ is the canonical isomorphism. By the first module isomorphism
    theorem we have
    \begin{equation}
        \notag
        A \otimes M / \ker(g) \simeq \im(g).
    \end{equation}
    Since the sequence is exact, $g$ is injective, hence $\im(g) = \left( A /
    \mathfrak{a} \right) \otimes M$. We identify $\ker(g)$ by the elements in
    $(\psi \circ f)(\mathfrak{a} \otimes M)$. Hence,
    \begin{equation}
        \notag
        A \otimes M / \ker(g) \simeq M / \mathfrak{a}M \simeq \left( A /
        \mathfrak{a} \right) \otimes M
    \end{equation}
    as we wanted to show.
\end{exercise}

\begin{exercise}[3]
    Let $A$ be a local ring, and assume that $M$ and $N$ are finitely generated
    $A$-modules. Assume that $M \otimes_A N = 0$. We wish to show that then
    either $M = 0$, or $N = 0$. Define $k = A / \mathfrak{m}$ where
    $\mathfrak{m}$ is the only maximal ideal in $A$. Furthermore, let $M_k = k
    \otimes M$ and $N_k = k \otimes N$ be the $k$ modules obtained from $M$ and
    $N$ by extension of scalars.

    From the previous exercise, we know that $M_k \simeq M / \mathfrak{m}M$ and
    $N_k \simeq N / \mathfrak{m}N$. If we can show that either $M_k = 0$ or
    $N_k = 0$, then by Nakayama's lemma we have
    \begin{equation}
        \notag
        M_k = 0 \implies M / \mathfrak{m}M = 0 \implies M = \mathfrak{m}M
        \implies M = 0
    \end{equation}
    since $\mathfrak{m}$ is contained in the Jacobson radical of $A$ and $M$ is
    finitely generated (similarly for $N_k$).  In order to show either $M_k =
    0$ or $N_k = 0$ we consider the fact that $M \otimes_A N = 0$. This
    gives\footnote{I think we use the fact that direct sum and direct product
    coincide if the index set is finite, in order to apply proposition
    2.14(iii)}
    \begin{equation}
        \notag
        M \otimes_A N = 0 \implies (M \otimes_A N)_k = 0 \implies M_k \otimes_k
        N_k = 0.
    \end{equation}
    \textcolor{red}{Since $M_k \otimes_k N_k = 0$ it has dimension zero. The
    dimension of $M_k \otimes_k N_k$ is the product of the dimensions of $M_k$
    and $N_k$. However, this means that the dimension of either $M_k$ or $N_k$
is zero. Hence, either $M_k = 0$ or $N_k = 0$.} Therefore, by the above mention
of Nakayama's lemma, we have either $M = 0$ or $N = 0$.
\end{exercise}

\begin{exercise}[4]
    Let $\left\{ M_i \right\}_{i \in I}$ be family of $A$-modules, and let $M =
    \bigoplus_{i \in I}M_i$.  We wish to show that $M$ being flat is equivalent
    to each $M_i$ being flat.
    
    Assume that each $M_i$ is flat. Then by definition, if
    \[
        \begin{tikzcd}
            0 \arrow[r] & N' \arrow[r, "f"] & N \arrow[r, "g"] & N'' \arrow[r] & 0
        \end{tikzcd}
    \]
    is an exact sequence, then 
    \[
        \begin{tikzcd}
            0 \arrow[r] & N'\otimes M_i \arrow[r, "f_i"] & N \otimes M_i\arrow[r, "g _i"] & N''\otimes M_i \arrow[r]& 0
        \end{tikzcd}
    \] is an exact sequence. Let $\varphi \colon \bigoplus (N' \otimes M_i) \to
    \bigoplus (N \otimes M_i)$ be defined by
    \begin{equation}
        \notag
        \varphi(x) = \left( f_1(x_1), f_2(x_2), \ldots \right)
    \end{equation}
    and $\psi \colon \bigoplus (N \otimes M_i) \to \bigoplus (N'' \otimes M_i)$
    be defined by 
    \begin{equation}
        \notag
        \psi(y) = \left( g_1(y_1), g_2(y_2), \ldots \right).
    \end{equation}
    By the injectivity of each $f_i$ and the surjectivity of each $g_i$ we have
    that $\varphi$ is injective and $\psi$ surjective. Hence the sequence
    \[
        \begin{tikzcd}
            0 \arrow[r] & N' \otimes M \arrow[r, "\varphi"] & N \otimes M \arrow[r, "\psi"] & N'' \otimes M \arrow[r] & 0 
        \end{tikzcd}
    \]
    is exact. To show that $M$ flat implies $M_i$ flat, we argue by
    contradiction. Assuming $M$ flat implies that the above short sequence is
    exact. That is $\varphi$ and $\psi$ are injective and surjective
    respectively. If one of the $M_i$'s were not flat, this would contradict
    the fact that $\varphi$ is injective, or that $\psi$ is surjective.
    Consequently all the $M_i$'s must be flat.
\end{exercise}
\begin{exercise}[5]
    We let $A[x]$ be the ring of polynomials over the indeterminate $x$. We
    wish to show that $A[x]$ is a flat $A$-algebra. We proceed accordingly:
    \begin{enumerate}
        \item Show that $A$ is a flat $A$-module;
        \item show that $M_i = Ax^i \simeq A$;
        \item show that $\bigoplus_i M_i = A[x]$;
        \item apply exercise 2.4.
    \end{enumerate}
    
    To show that $A$ is a flat $A$-module we check that any exact sequence of
    $A$-modules is exact when tensored with $A$. Let $0 \to M' \to M \to M''
    \to 0$ be an exact sequence of $A$-modules. Tensoring yields
    \[
    \begin{tikzcd}
        0 \arrow[r] & M'\otimes A \arrow[r]\arrow[d, "\simeq"] &
        M\otimes A \arrow[r]\arrow[d, "\simeq"] & M''\otimes A \arrow[r] \arrow[d, "\simeq"] & 0 \\
        0 \arrow[r] & M' \arrow[r] & M \arrow[r] & M'' \arrow[r] & 0
    \end{tikzcd}
    \]
    This sequence is exact, hence $A$ is a flat $A$-module.  Define now $M_i =
    Ax^i$. Under the map $f_i : M_i \to A$ defined by $ax^i \mapsto a$ we see
    that $M_i \simeq A$, therefore each $M_i$ is flat. It then follows that any
    element in $A[x]$ can be written as a sum of elements in the $M_i$'s. That
    is,
    \begin{equation}
        \notag
        \underset{i}{\bigoplus}M_i = A[x].
    \end{equation}
    Applying exercise 2.4, we know that since each $M_i$ is flat, that $A[x]$
    is flat. 
\end{exercise}

\begin{exercise}[7]
    Let $\mathfrak{p}$ be a prime ideal in a ring $A$. We want to show that
    $\mathfrak{p}[x]$ is a prime ideal in $A[x]$. We can do this by showing
    that $A[x] / \mathfrak{p}[x]$ is an integral domain. Consider the canonical
    homomorphism
    \begin{equation}
        \notag
        \phi : A[x] \to \left(A / \mathfrak{p}\right)[x].
    \end{equation}
    This is surjective with kernel $\mathfrak{p}[x]$. Hence by the first
    isomorphism theorem for modules we have
    \begin{equation}
        \notag
        A[x] / \mathfrak{p}[x] \simeq \left( A / \mathfrak{p} \right)[x].
    \end{equation}
    Since $A / \mathfrak{p}$ is an integral domain, the polynomial ring with
    coefficients in this integral domain must be an integral domain. Hence,
    $\mathfrak{p}[x]$ is a prime ideal in $A[x]$.  This implication does not
    hold if we replace \emph{prime} with \emph{maximal}. Take for instance the
    maximal ideal $2\mathbb{Z}$ in $\mathbb{Z}$. The quotient $\mathbb{Z} /
    2\mathbb{Z}$ is a field. However, $x \in
    \left(\mathbb{Z}/2\mathbb{Z}\right)[x]$ does \emph{not} have a
    multiplicative inverse.
\end{exercise}

\begin{exercise}[8]
    We want to show the following:
    \begin{enumerate}[i)]
        \item If $M$ and $N$ are $A$-flat, then so is $M \otimes_A N$;
        \item If $B$ is a flat $A$-algebra and $N$ is a flat $B$-module, then
            $N$ is $A$-flat as an $A$-module.
    \end{enumerate}
    Consider first $M$,$N$ as $A$-flat modules. Using the isomorphism from
    proposition 2.14(ii). Then tensoring any exact sequence by $M \otimes_A N$
    keeps the sequence exact as $M$ and $N$ are flat.
    
    For (ii) we want to use the injectivity characterization of flatness. Let
    $\varphi$ be any injective map from $M'$ to $M$, with $M$, $M'$ both being
    $A$-modules. Since $B$ is a flat $A$-module we know that 
    \begin{equation}
        \notag
        \varphi \otimes \id_B \colon M' \otimes_A B \to M \otimes_A B
    \end{equation}
    is injective. We know by assumption that $N$ is a flat $B$-module, so 
    the map
    \begin{equation}
        \notag
        \left( \varphi \otimes \id_B \right) \otimes \id_N \colon \left(M'
        \otimes_A B \right) \otimes_B N \to \left( M \otimes_A B \right)
        \otimes_B N
    \end{equation}
    is injective and this makes sense when regarding $M \otimes_A B$ as a
    $B$-module (by exercise 2.15). By proposition 2.14(iv) this is
    isomorphic to
    \begin{equation}
        \notag
        \left( \varphi \otimes \id_B \right) \otimes \id_N \colon M' \otimes_A N \to M \otimes_A N.
    \end{equation}
    This map is the map we want as the following diagram commutes:
    \[
        \begin{tikzcd}
            (m' \otimes_A b) \otimes_B n \ar[rr]\ar[d] & &\left( \varphi(m') \otimes_A b\right) \otimes_B n \ar[d] \\
            m' \otimes_A (b \otimes_B n) \ar[d] & \circlearrowleft & \varphi(m') \otimes_A (b \otimes_B n) \ar[d] \\
            m' \otimes_A bn \ar[rr] & & \varphi(m')\otimes_A bn
        \end{tikzcd}
    \]
    This map being injective means that $N$ is flat as an $A$-module.
\end{exercise}

\begin{exercise}[8]
    We want to show that if $M''$ and $M'$ are finitely generated and $0 \to M'
    \to M \to M'' \to 0$ is exact, then it follows that $M$ is finitely
    generated. We show the special case that $M'$ is a submodule of $M$ and
    that $M'' = M / M'$. Let $M'$ have generating set $\left\{ x_i \right\}$
    and $M''$ generating set $\left\{ y_i \right\}$.
\end{exercise}

\begin{exercise}[10]
    Let $A$ be a ring, $\mathfrak{a}$ an ideal contained in the Jacobson
    radical $\mathfrak{R}$. Let $M$ be an $A$-module and $N$ a finitely
    generated $A$-module. We want to show that $u \colon M \to N$ is
    surjective if the induced map $v \colon M /\mathfrak{a}M \to N
    /\mathfrak{a}N$ is surjective. Seeing the words \emph{finitely
    generated} and \emph{Jacobson radical} we want to apply Nakayama's
    lemma.
    
    Consider $\coker(u) = N / \im{u}$. If we can show this to be zero,
    then $\im{u} = u(M) = N$ so $u$ is surjective. Tensoring the exact
    sequence $M \to N \to \coker(u) \to 0$ with $(A/\mathfrak{a})$ yields
    \begin{equation}
        \notag
        \begin{tikzcd}
            M \otimes (A / \mathfrak{a}) \ar[r, "u\otimes \id"] \ar[d, "\simeq"] & N \otimes (A / \mathfrak{a}) \ar[r]  \ar[d, "\simeq"]& \coker(u) \otimes (A / \mathfrak{a}) \ar[r] \ar[d, "\simeq"] & 0 \\
            M / \mathfrak{a}M \ar[r, "v"] & N / \mathfrak{a}N \ar[r] & \coker(u) / \mathfrak{a}\coker(u) \ar[r] & 0
        \end{tikzcd}
    \end{equation}
    The fact that $u\otimes \id = g$ took a while to figure out, but it
    happens in the abstract nonsense from exercise 2.2. Consequently
    $\coker(u) = \mathfrak{a}\coker(u)$, and since $\coker(u)$ is finitely
    generated and $\mathfrak{a}$ is contained in $\mathfrak{R}$, it
    follows from Nakayama's lemma that $\coker(u) = 0$ which implies that
    $u(M) = N$, hence $u$ is surjective.

    \textcolor{blue}{An equivalent proof using a corollary of Nakayama is way
    easier.} Consider the composition $f\colon M \to M/\mathfrak{a}M \to
    N/\mathfrak{a}N$. This is surjective by assumption. We now examine the
    image of $f$ in two different ways. First of all we have $\im(f) = N / \a
    N$ using the surjectivity. We also have that $\im(f) = \im(u)/\a N$. Now,
    since we are modding out by $\a N$, we can simply add the submodule $\a N$
    to obtain $\im(f) = (\im(u) + \a N) / \a N$ in order to get something on
    the form of Nakayama. So we have $N / \a N = (\im(u) + \a N) / \a N$, so $N
    = \im(u) + \a N)$. By corollary 2.7 we have that $\im(u) = N$ hence $u$ is
    surjective.
\end{exercise}

\subsection{Chapter 3: Rings and modules of fractions}
\label{sub:chapter_3}

\begin{exercise}[1]
    Given a ring $A$, a multiplicatively closed subset $S$ of $A$ and a
    finitely generated $A$-module $M$, we want to show that $S^{-1}M = 0$ if and
    only if there exists an $s \in S$ such that $sM = 0$.

    Let $M$ have generators $x_1, \ldots, x_n$. Assume first that $S^{-1}M = 0$.
    This means that $m / s = 0 / 1$ in $S^{-1}M$ for all $m \in M, s \in S$. This
    in turn, means that by definition there exists $t$ in $S$ such that $tm =
    0$ in $M$. In particular, let $t_i$ be the element such that $tx_i = 0$,
    i.e., an element of $S$ that kills one of the generating elements of $M$.
    Since $S$ is multiplicatively closed, take $t' = t_1t_2\ldots t_n$. Then
    $t'$ kills all generators of $M$. Hence, $t'm = t' \sum^{}_{} a_i x_i = 0$
    and consequently, $t'M = 0$.

    For the converse, assume that there exist an $s \in S$ such that $sM = 0$.
    Consider the element $m / 1$ in $S^{-1}M$. Then
    \begin{equation}
        \notag
        \frac{m}{1} = \frac{sm}{s} = \frac{0}{s} = \frac{0}{1},
    \end{equation}
    so any element $m / 1$ in $S^{-1}M$ is zero, hence $S^{-1}M = 0$.
\end{exercise}

\begin{exercise}[2]
    Let $\mathfrak{a}$ be an ideal of a ring $A$. Furthermore, let $S = 1 +
    \mathfrak{a}$. We want to show that $\loc{S}{\mathfrak{a}}$ is contained in
    the Jacobson radical $\mathfrak{R}$ of $\loc{S}{A}$.

    We first show that $S$ is infact a multiplicatively closed subset of $A$.
    We first require $1$ to be in $S$, but since $0$ is contained in any ideal,
    we have $1 + 0 = 1 \in S$. Taking two elements $(1 + a)$ and $(1 + b)$ in
    $S$ we see that $(1 + a + b + ab)$ is in $S$ as $\mathfrak{a}$ is an ideal.
    Consequently, $S$ is a mulitplicative set.

    In order to check whether $\loc{S}{\mathfrak{a}}$ is contained in the
    Jacobson, we pick an element $x$ in $\loc{S}{\mathfrak{a}}$ and show that
    $1 - xy$ is a unit in $\loc{S}{A}$ for all $y$ in $\loc{S}{A}$. We have that
    \begin{equation}
        \notag
        1 - xy = \frac{1}{1} - \frac{aa'}{(1 + b)(1 + b')} = \frac{(1 + b)(1 + b') - aa'}{(1 + b)(1 + b')}
    \end{equation}
    the numerator is on the form $(1 + a'')$ where $a''$ is in $\mathfrak{a}$,
    and the denominator is on the form $(1 + b'')$ with $b''$ in
    $\mathfrak{a}$. Consequently, both numerator and denominator are units in
    $\loc{S}{A}$, so $1 - xy$ is a unit.

    We now wish to use this fact along with Nakayama's lemma in order to prove
    corollary 2.5 without the use of determinants. So, let $M$ be finitely
    generated with $\mathfrak{a}$ an ideal such that $\mathfrak{a}M = M$.
    We localize in $S = 1 + \mathfrak{a}$ which yields
    \begin{equation}
        \notag
        \loc{S}{M} = \loc{S}{\mathfrak{a}M} = \left(\loc{S}{\mathfrak{a}}\left)\left( \loc{S}{M} \right).
    \end{equation}
    Now, from the previous result we know that $\loc{S}{\mathfrak{a}}$ is
    contained in $\mathfrak{R}$. Since we also have that $\loc{S}{M}$ is
    finitely generated, due to $M$ being finitely generated, we can conclude
    from Nakayama's lemma that $\loc{S}{M} = 0$. By exercise 1 we know that
    this holds if and only if there exist an $s \in S$ such that $sM = 0$.
    This $s$ is on the form $s = 1 + a$ for some $a$ in $\mathfrak{a}$, hence
    $s \equiv 1 \pmod{\mathfrak{a}}$.
\end{exercise}

\begin{exercise}[3]
    Let $A$ be a ring, and let $S$ and $T$ be two multiplicatively closed
    subsets of $A$. Let $U$ be the image of $T$ in $\loc{S}{A}$. We want to
    show that the rings $\loc{(ST)}{A}$ and $\loc{U}{\left( \loc{S}{A}
    \right)}$ are isomorphic. For brevity, we shall call these rings $B$ and
    $C$ respectively. We let $f$ be the only sensible choice of map between
    these two sets and show that it is a bijection. That is $f \colon
    \loc{(ST)}{A} \to \loc{U}{\left( \loc{S}{A} \right)}$ defined by
    \begin{equation}
        \notag
        f\left(a / (st)\right)  = (a / s) / (t / 1).
    \end{equation}

    We first show that $f$ is well defined. Take $a / (st) = a' / (s't')$ and
    consider the image of these two elements. We need to find a $u$ in $C$ such
    that $u \left( at' / s \right) = u \left( ta' / s' \right)$. But, by
    assumption there exists a $u'$ in $B$ such that $u'as't' = u'a'st$, so we
    can chose $u = u' / 1$. Consequently, $f$ is a well defined map.

    We now need to show that $f$ is a bijection. We see immediately that it is
    surjective, since any element on the form $(a / s) / (t / 1)$ comes from an
    element on the form $a / (st)$ with $a \in A$ and $st \in ST$. To see that
    it is injective, take an element $a / (st)$ in the kernel of $f$. Then $(a
    / s)/(t / 1) = (0/1) / (1 / 1)$, so there is a $u$ such that
    \begin{align*}
        u\left((a / s)(1 / 1) -  (t / 1)(0 / 1)\right) = 0,
    \end{align*}
    so $u(a / s) - (0 / 1) = 0$, but this means that $a / st = 0$ in $B$. So
    $f$ is injective. It remains to show that $f$ is a ring homomorphism. 
\end{exercise}

\subsection{Chapter 4: Primary Decomposition}
\label{sub:chapter_4_primary_decomposition}

\begin{exercise}[2]
    We wish to show that if an ideal $\a$ satisfies $\rad{\a} = \a$, then it
    has no embedded prime ideals. Since we are dealing with ideals having a
    minimal primary decomposition, we this be
    \begin{equation}
        \notag
        \a = \bigcap_{i=1}^n \q_i.
    \end{equation}
    We denote the radical $\rad{\q_i}$ by $\p_i$. Now, since $\a = \rad{a}$, we
    know that $\a = \bigcap \p_j$ where $p_j$ is a prime ideal containing $\a$.
    Taking radicals we get that, on one hand,
    \begin{equation}
        \notag
        \a = \rad{\a} = \rad{\bigcap \p_j} = \bigcap \rad{\p_j} = \bigcap \rad{\p_j} = \bigcap \p_j,
    \end{equation}
    while on the other, 
    \begin{equation}
        \notag
        \rad{\a} = \bigcap_{i=1}^n \rad{q_i} = \bigcap_{i=1}^n
        p_i.
    \end{equation}
    \textcolor{red}{I now need to properly figure out exactly \emph{why} this
    means that none of the prime ideals belonging to $\a$ are embedded.}
\end{exercise}

\begin{exercise}[4]
    We consider the polynomial ring $\Z[t]$. We want to show that the ideal
    $\mathfrak{m} = (2, t)$ is maximal, and that the ideal $\q = (4, t)$ is
    $\m$ primary, but \emph{not} a power of $\m$.  Note first that the quotient
    $\Z[t] / \m$ consists of all constant polynomials where the constant is
    either 0 or 1. Hence, this is isomorphic to the field $\Z_2$. Consequently,
    $\m$ is maximal. In order to show that $\q$ is $\m$ primary, we first show
    that all the zero-divisors in $\Z[t] / \q$ are nilpotent, and then show
    that the radical $\q$ is equal to $\m$. The quotient $\Z[t] / \q$ is
    isomorphic to $\Z_4$, and consists therefore only of constant polynomials
    where the constant is either 0, 1, 2 or 3. The only zero-divisor in $\Z_4$
    is 2, and this is also nilpotent, as $2^2 = 0$. Consequently, $\q$ is
    primary. Consider now $\rad{\q}$:
    \begin{equation}
        \notag
        \rad{\q} = \rad{(4, t)} = \rad{(4) + (t)} = \rad{\rad{(4)} + \rad{(t)}} = \rad{(2) + (t)} = (2) + (t)
    \end{equation}
    since $(2) + (t)$ is maximal (and therefore also prime). This shows that
    $\q$ is $\m$-primary.  \textcolor{red}{It remains to show that $\q$ is
    \emph{not} a power of of $\m$.}
\end{exercise}

\begin{exercise}[5]
        
\end{exercise}

\section{Problem Sheet}
\label{sec:problem_sheet}
\begin{problem}[1]
    We let $M$ be an $A$ module, and let $N$ be a submodule of $M$. We define
    the radical of $N$ in $M$ as
    \begin{equation}
        \notag
        r_M(N) = \left\{ x \in A \mid x^n \subseteq N \text{ for some } n > 0\right\}.
    \end{equation}
    \begin{enumerate}[a)]
        \item We first want to show that $r_M(N) = \rad{(N : M)}$. Recall that
            $(N  : M)$ is the set of all $x \in A$ that multiplies $M$ into
            $N$. Let $x \in r_M(N)$. Then $x^n m \in N$ for all $m \in M$.
            Hence $x^n \in (N  : M)$, so $x \in \rad{(N : M)}$.  Conversely,
            let $x \in \rad{(N:M)}$, then $x^n \in (N:M)$, so $x^nM \subseteq
            N$, hence $x \in r_M(N)$.
        \item We now wish to show that $(Q : M)$ is a primary ideal in $A$
            under the assumption that $Q$ is primary in $M$ as a module. Let
            $ab \in (Q : M)$, and assume in addition that $a \notin (Q : M)$.
            We need to show that $b^n \in (Q : M)$ for some $n > 0$. By
            assumption, $a \notin (Q : M)$ tells us that there is some element
            $m' \in M$ such that $am' \notin Q$. We know in addition, that
            $bam' \in Q$ since $ab \in (Q : M)$. 

            Consider now the map $\varphi_b : M \to M$ given by $\varphi_b(x) =
            xb$.  We have for the element $am' \notin Q$ that $\varphi_b(am')$
            is in $Q$. Hence, in the quotient $M / Q$, we map a non-zero
            element to a zero element, so the induced map $\bar{\varphi_b}$ has
            non-zero kernel, hence not injective. This means that $\bar{b}$ is
            a zero-divisor in $M / Q$, and consequently, since $Q$ is primary
            in $M$, $b$ is nilpotent. This means that there is an $n > 0$, such
            that $\varphi(b)^n = 0$, hence $b^nm \in Q$ for all $m$. We can
            therefore conclude that $b^n \in (Q : M)$.
\end{problem} 
\end{document}
